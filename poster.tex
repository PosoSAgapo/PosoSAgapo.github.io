\documentclass[a0paper,portrait]{baposter}

\usepackage{times}
\usepackage{calc}
\usepackage{url}
\usepackage{graphicx}
\usepackage{amsmath}
\usepackage{amssymb}
\usepackage{relsize}
\usepackage{multirow}
\usepackage{booktabs}
\usepackage{rotating}
\usepackage{bm}
\usepackage{enumitem}
\usepackage[utf8]{inputenc}
\usepackage[T1]{fontenc}
\usepackage{subcaption}
\usepackage{adjustbox}
\usepackage{xcolor}

% Define colors
\definecolor{darkblue}{RGB}{0,51,102}
\definecolor{lightblue}{RGB}{173,216,230}
\definecolor{orange}{RGB}{255,165,0}
\definecolor{green}{RGB}{34,139,34}

\begin{document}

\begin{poster}{
  % Poster Options
  grid=false,
  columns=2,
  colspacing=0.35em,
  boxpadding=0.5em,
  headerheight=0.1\textheight,
  headershape=roundedright,
  headerfont=\Large\bf\textsc,
  textborder=roundedleft,
  boxshade=none,
  borderColor=darkblue,
  headerColorOne=darkblue,
  headerColorTwo=lightblue,
  headerFontColor=white,
  boxColorOne=white,
  boxColorTwo=white,
  headershade=shadelr,
  background=none,
  eyecatcher=true
}
% Empty space for logo
{}
% Title
{\Large\bf\textsc{A Statistical and Multi-Perspective Revisiting of the\\Membership Inference Attack in Large Language Models}\vspace{0.2em}}
% Authors
{\normalsize\textsc{Bowen Chen$^{1}$, Namgi Han$^{1}$, Yusuke Miyao$^{1,2}$}\vspace{0.15em}\\
\small $^1$Department of Computer Science, The University of Tokyo \quad $^2$Research and Development Center for Large Language Models, National Institute of Informatics\\
\small\texttt{\{bwchen, hng88, yusuke\}@is.s.u-tokyo.ac.jp}
}
% Institution logos
{\begin{tabular}{l}
\includegraphics[height=2em]{figures/logo1.png} \\
\includegraphics[height=2em]{figures/logo2.png} \\
\includegraphics[height=2em]{figures/logo3.png}
\end{tabular}}

%%%%%%%%%%%%%%%%%%%%%%%%%%%%%%%%%%%%%%%%%%%%%%%%%%%%%%%%%%%%%%%%%%%%%%%%%%%%%%
\headerbox{Motivation \& Experimental Setup}{name=abstract,column=0,row=0}{

\begin{minipage}[c]{0.5\textwidth}
\centering
\includegraphics[width=0.8\linewidth]{figures/newfigure.pdf}
\end{minipage}
\begin{minipage}[c]{0.5\textwidth}
\textbf{Problems:}
\begin{itemize}[leftmargin=em,itemsep=0.05em]
\item Single-setting MIA experiments → biased evaluation, creating an inconsistent performance across previous literature.
\item We need a statistical-level analysis
\end{itemize}
\textbf{Setup:} 5,000+ experiments to profile one MIA method

\begin{itemize}[leftmargin=em,itemsep=0.05em]
\item \textbf{Methods}: 11 MIA methods
\item \textbf{Models}: Pythia series, OLMo2 series
\end{itemize}


\end{minipage}
}

%%%%%%%%%%%%%%%%%%%%%%%%%%%%%%%%%%%%%%%%%%%%%%%%%%%%%%%%%%%%%%%%%%%%%%%%%%%%%%
\headerbox{Statistical Analysis Results}{name=results,column=0,span=2,below=abstract,aligned=overlap}{
%%%%%%%%%%%%%%%%%%%%%%%%%%%%%%%%%%%%%%%%%%%%%%%%%%%%%%%%%%%%%%%%%%%%%%%%%%%%%%
\begin{minipage}[c]{0.55\textwidth}
\centering
\includegraphics[width=1\linewidth, height=0.4\linewidth]{figures/comparison_subplots.png}
\end{minipage}
\hfill
\begin{minipage}[c]{0.42\textwidth}
\begin{itemize}[leftmargin=1em,itemsep=0.1em]
\item \textbf{Splitting Method Effect}: Relative Split > Complete Split > Truncate Split
\item \textbf{Model Size}: Performance scales with model size, indicating an internal change occurred in LLM that benefits the MIA when increasing model size.
\item \textbf{Domain Differences}: Wikipedia, FreeLaw perform best, e.g, free text domain with high token diversity; code/math domains worse, e.g, domains with lower token diversity
\item \textbf{Method Comparison}: Min-k\% ++ and ReCaLL relatively better, but most methods don't significantly outperform baselines 
\end{itemize}
\end{minipage}
}

\headerbox{Threshold Stability}{name=outliers,column=0,span=2,below=results}{
\begin{minipage}[c]{0.5\textwidth}
\centering
\includegraphics[width=1\linewidth, height=0.45\linewidth]{figures/corped_boxplot_best_threshold.png}
\end{minipage}
\hfill
\begin{minipage}[c]{0.42\textwidth}
\begin{itemize}[leftmargin=1em,itemsep=0.1em]
\item Thresholds vary significantly across domains, and outliers exist, showing the instability of a good threshold.
\item Thresholds change with model sizes, meaning that a threshold decided in a small or large model cannot be applied to another model size
\item The lack of universal thresholds is an overlooked challenge for MIA methods in real-world applications, severely affecting practicality.
\end{itemize}

\end{minipage}
}



%%%%%%%%%%%%%%%%%%%%%%%%%%%%%%%%%%%%%%%%%%%%%%%%%%%%%%%%%%%%%%%%%%%%%%%%%%%%%%
\headerbox{Method Overlap Analysis}{name=overlap,column=1,row=0}{
%%%%%%%%%%%%%%%%%%%%%%%%%%%%%%%%%%%%%%%%%%%%%%%%%%%%%%%%%%%%%%%%%%%%%%%%%%%%%%
\begin{minipage}[c]{0.5\textwidth}
\centering
\includegraphics[width=1\linewidth]{figures/overlap_matrix.pdf}
\end{minipage}
\hfill
\begin{minipage}[c]{0.5\textwidth}
\begin{itemize}[leftmargin=1em,itemsep=0.05em]
\item Different MIA methods differentiate different MIA datasets 
\item No single method dominates other MIA methods.
\item This suggests that different MIA methods are effective in different situations.
\end{itemize}

\end{minipage}
}

%%%%%%%%%%%%%%%%%%%%%%%%%%%%%%%%%%%%%%%%%%%%%%%%%%%%%%%%%%%%%%%%%%%%%%%%%%%%%%
\headerbox{Embedding Analysis}{name=embedding,column=0,span=2,below=outliers}{
%%%%%%%%%%%%%%%%%%%%%%%%%%%%%%%%%%%%%%%%%%%%%%%%%%%%%%%%%%%%%%%%%%%%%%%%%%%%%%
\begin{minipage}[c]{0.55\textwidth}
\centering
\includegraphics[width=1\linewidth, height=0.5\linewidth]{figures/all_models_embedding_result.png}
\end{minipage}
\hfill
\begin{minipage}[c]{0.42\textwidth}
\begin{itemize}[leftmargin=1em,itemsep=0.1em]
\item \textbf{Internal Correlation}: Whether being differentiable or not is also reflected in the embedding level.
\item \textbf{Model Size}: Large models show emergent behavior, significantly improved embedding separability when reaching deep layers in larger models.
\item \textbf{Final Layer Limitation}: Current MIA methods' reliance on final layer embeddings is suboptimal, as the final layer has a low embedding separability.
\end{itemize}


\end{minipage}
}

%%%%%%%%%%%%%%%%%%%%%%%%%%%%%%%%%%%%%%%%%%%%%%%%%%%%%%%%%%%%%%%%%%%%%%%%%%%%%%
\headerbox{Conclusions \& Future Work}{name=conclusion,column=0,span=2,below=embedding,above=bottom}{
%%%%%%%%%%%%%%%%%%%%%%%%%%%%%%%%%%%%%%%%%%%%%%%%%%%%%%%%%%%%%%%%%%%%%%%%%%%%%%
\begin{minipage}[c]{0.65\textwidth}
\begin{itemize}[leftmargin=1em,itemsep=0.1em]
\item We conducted a statistical revisiting of the MIA method to address the performance inconsistency.
\item It is challenging to determine a stable threshold for differentiating between members and non-members.
\item Different MIA methods work in different situations, indicating that there are no "winner-takes-all" situations in MIA.
\item The differentiability is reflected in LLM's inner structure, including both the embedding separability and the decoding dynamics.
\end{itemize}


% \textbf{Future Directions:}
% \begin{itemize}[leftmargin=1em,itemsep=0.1em]
% \item Develop adaptive MIA method combination strategies
% \item Explore integration of multi-layer embedding information
% \item Address threshold stability issues
% \item Extend evaluation to larger-scale models
% \end{itemize}
\end{minipage}
\hfill
\begin{minipage}[c]{0.32\textwidth}
\begin{center}
\colorbox{lightblue!20}{\begin{minipage}{0.9\textwidth}
\centering
\vspace{0.3em}
{\small\textbf{\textcolor{darkblue}{GitHub Repository}}}\\
\vspace{0.2em}
\includegraphics[width=0.6\textwidth]{figures/qr_code.png}\\
\vspace{0.1em}
{\tiny\textcolor{gray}{Scan QR code to access GitHub repository}}\\
\vspace{0.3em}
\end{minipage}}
\end{center}
\end{minipage}
}

\end{poster}
\end{document} 